\chapter{Material parameters}
In the simulations the units millimeters and grams are used. This combination gives back the SI-unit $\frac{N}{m^2} = \frac{kg\,\,m}{s^2 m^2} = \frac{g}{s^2\,mm} = Pa$ for pressure, and is also convenient when considering the scale of the spinal cord. CSF consists of 99\% water \cite{Fish92}, and thus CSF is modeled as water at $37^o C$, i.e $\rho_f$ = $10^{-3} \frac{g}{mm^3}$ and $\nu_f = 0.658 \frac{mm^2}{s}$.
\\
\\
For the spinal cord, studies have shown a huge variety in material parameters. One of the most measured properties in the mammalian central nervous system is probably the Young's modulus, $E$ according to Smith, Humphrey \cite{Smit07}. In addition to this, values for the Poisson ratio, $\nu_P$ needs to be found. In the literature, there is a huge gap in reported Young's modulus for spinal cord tissue. These studies does not distinguish between grey and white matter in the spinal cord, and neither will we. In general, grey and white matter will have different elastic and porous properties, however as shown by St{\o}verud et al (2015) \cite{Stov15}, the distinction between the two have shown to have negligible effect except for in small local regions depending on patient-specific distribution of the two. Developing patient-specific models are highly relevant for a precise description of the CSF flow, but is not the main goal for this thesis. 


\begin{table}[!ht]
  \begin{center}
  \begin{tabular}{l  l  l  l  l  l}
	\hline
    Article & Region &  Model & Parameters \\ \hline
	Hung et al. \cite{Hung81} & spinal cord & Experimental & E = 0.26 MPa  \\ \\
	Ben-Hatira et al. \cite{Ben12} & spinal cord & Nonlinear elastic & E  = 1.4 MPa \\
	& & &  $\nu $  = 0.499 \\
	Ozawa et al. \cite{Ozaw04} & spinal cord & Experimental & E = 16 kPa \\ \\
	Smith and Humphrey \cite{Smit07} & brain & Experimental & E = 5.0 kPa \\
	& & &  $\nu$ = 0.479 \\
	Cheng et al. \cite{Chen08} & spinal cord & Review & E = 0.0119-1.98 MPa \\ \\
	Clarke \cite{Clar10} & spinal cord & Review & E = 0.012-1.37 MPa \\ \\
	Persson et al. \cite{Pers10} & spinal cord & Review  & E = 0.26-1.3MPa \\
	& & (linear elastic) & \\ 
    \hline 
  \end{tabular}
  \caption{Summary of elastic parameters used in literature (as presented in Klystad \cite{Kyls14})}
  \end{center}
\end{table}
From this, Lame's parameters for the spinal cord were determined as
\[ \mu_s = \frac{E}{2(1 + \nu_P)} \] and
\[ \lambda_s = \frac{\nu_P E}{(1 + \nu_P)(1-2\nu_P)} = 3.9 \cdot 10^4 \text{ Pa} \]
The spinal cord has fibres oriented in the axial direction and a direction-dependent Youngs modulus would then be expected. In the literature values range between 0.012 to 1.98 MPa. As reported by \cite{Clar10} most of spinal cord experimental studies use a tensile test, and the stress-strain and stress-relaxation responses of the spinal cord are non-linear. Therefore, it will in general be hard to compare results from different studies using different arbitrary levels of strain. Another approach used by Kwon \cite{Kwon02}, focuses more on spinal cord injuries and are thus more interested in properties during compression. From the modeling point of view, both approaches is of interest, and better constitutive models could be attained by combining results from several of these studies. In this work, however, we limit Youngs modulus to be a constant independent of spatial direction. For the spinal cord, this assumption has been made in all the previous cited works in this thesis. 
\\
\\
The permeability, $\kappa$ is used as a measurement for the how fluid flows in a porous medium. A large permeability indicates a pervious medium. We use the value from [Karen, Ida]
\[ \kappa = 1.4\cdot 10^{-15} m^2 \]

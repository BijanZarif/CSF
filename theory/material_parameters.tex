\chapter{Material parameters}
In the simulations the units millimeters and grams are used. This combination gives back the SI-unit $\frac{N}{m^2} = \frac{kg\,\,m}{s^2 m^2} = \frac{g}{s^2\,mm} = Pa$ for pressure, and is also convenient when considering the scale of the spinal cord. CSF is modeled as water at $37^o C$, i.e $\rho_f$ = $10^{-3} \frac{g}{mm^3}$ and $\nu_f = 0.658 \frac{mm^2}{s}$.
\\
\\
For the spinal cord, studies have shown a huge variety in material parameters. One of the most measured properties in the mammalian central nervous system is probably the Young's modulus, $E$ according to Smith, Humphrey \cite{Smit07}. In addition to this, values for the Poisson ratio, $\nu_P$ needs to be found. In the literature, there is a huge gap in reported Young's modulus for spinal cord tissue. 


\begin{table}[!ht]
  \begin{center}
  \begin{tabular}{l  l  l  l  l  l}
	\hline
    Article & Region &  Model & Parameters \\ \hline
	Hung et al. \cite{Hung81} & spinal cord & Experimental & E = 0.26 MPa  \\ \\
	Ben-Hatira et al. \cite{Ben12} & spinal cord & Nonlinear elastic & E  = 1.4 MPa \\
	& & &  $\nu $  = 0.499 \\
	Ozawa et al. \cite{Ozaw04} & spinal cord & Experimental & E = 16 kPa \\ \\
	Smith and Humphrey \cite{Smit07} & spinal cord & Experimental & E = 5.0 kPa \\
	& & &  $\nu$ = 0.479 \\
	Cheng et al. \cite{Chen08} & spinal cord & Review & E = 0.0119-1.98 MPa \\ \\
	Clarke \cite{Clar10} & spinal cord & Review & E = 0.012-1.37 MPa \\ \\
	Persson et al. \cite{Pers10} & spinal cord & Review (linear elastic) & E = 0.26-1.3MPa \\ 
    \hline 
  \end{tabular}
  \caption{Summary of elastic parameters used in literature (as presented in Klystad \cite{Kyls14})}
  \end{center}
\end{table}



From this, Lame's parameters for the spinal cord were determined as
\[ \mu_s = \frac{E}{2(1 + \nu_P)} \] and
\[ \lambda_s = \frac{\nu_P E}{(1 + \nu_P)(1-2\nu_P)} = 3.9 \cdot 10^4 \text{ Pa} \]
The permeability, $\kappa$ is used as a measurement for the how fluid flows in a porous medium. A large permeability indicates a pervious medium. We use the value from [Karen, Ida]
\[ \kappa = 1.4\cdot 10^{-15} m^2 \]

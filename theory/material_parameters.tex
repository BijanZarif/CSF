\chapter{Material parameters}
In the simulations the units millimeters and grams are used. This combination gives back the SI-unit $\frac{N}{m^2} = \frac{kg\,\,m}{s^2 m^2} = \frac{g}{s^2\,mm} = Pa$ for pressure, and is also convenient when considering the scale of the spinal cord. CSF is modeled as water at $37^o C$, i.e \\
\begin{tabular}{ l | l | l} \\
			\text{parameter} & \text{value} & \text{unit} \\ \hline
            $\rho_f$  & $10^{-3}$ & $\frac{g}{mm^3}$ \\ \hline
			$\nu_f$   & $0.658$ & $\frac{mm^2}{s}$ \\ \hline
\end{tabular}

For the spinal cord, studies have shown a huge variety in material parameters. One of the most measured properties in the mammalian central nervous system is probably the Young's modulus, $E$. [Smith, Humphrey 2006]. In addition to this, values for the Poisson ratio, $\nu_P$ needs to be found. Smith and Huphrey used the following values for these parameters. 
\[ E = 5*10^4 \text{dyn}/cm^2  = 5000 \text{ Pa}\]
\[ \nu_P = 0.479 \]
From this, Lame's parameters for the spinal cord were determined as
\[ \mu_s = \frac{E}{2(1 + \nu_P)} = 1.7 \cdot 10^3 \text{ Pa}\]
\[ \lambda_s = \frac{\nu_P E}{(1 + \nu_P)(1-2\nu_P)} = 3.9 \cdot 10^4 \text{ Pa}\]
The permeability, $\kappa$ is used as a measurement for the how fluid flows in a porous medium. A large permeability indicates a pervious medium. We use the value from [Karen, Ida]
\[ \kappa = 1.4\cdot 10^{-15} m^2 \]

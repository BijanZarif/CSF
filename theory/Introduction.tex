\chapter{Introduction}
The Cerebrospinal Fluid (CSF) surrounds the brain and acts as a protection to the brain inside the skull. As a result of the cardiac cycle, the CSF will flow up and down the subarachnoid space (SAS) surrounding the spinal cord. The Chiari malformation is a downwards displacement of a part of the brain known as the cerebellar tonsils that partially blocks CSF flow entering and leaving the SAS. This malformation is associated with syringomyelia, which is the presence of a fluid filled cavity within the spinal cord tissue. Treatment may include decompression surgery to remove parts of the bones of the skull to relieve pressure. Studies (e.g. Paul et al. (1983) \cite{Paul83}, Lorenzo et al. (1995) \cite{Lore95}, Guo et al. (2007)\cite{Guo07}) have shown that in many cases the syrinx gradually vanishes after surgery. The underlying mechanisms behind neither the formation nor the vanishing of the syrinx are not yet fully understood. 
\\
\\
\textit{In vivo} measurements by Quigley et al. (2004) \cite{Quig04} and Haughton et al. (2003) \cite{Haug03} have shown that the Chiari malformation is associated with abnormal CSF flow. Many researchers have thus suggested computational fluid dynamics (CFD) as a tool to give useful insight, as experiments are very difficult and expensive. These models have preticted abnormal CSF flow due to tonsilllar hernitation (Linge et al. (2011) \cite{Ling11}) as well as normalization of flow patterns modeling cases of post-operative craniovertebra decompression surgery. (Linge et al. (2014) \cite{Ling14})
\textit{In vitro} fluid-structure interaction (FSI) models by Martin et al. \cite{Mart09IV} have suggested that the presence and location of the syrinx have a critical impact on the pressure environment. Bertram et al. \cite{Bert09} have proposed in a series of \textit{in silico} studies using FSI, that the so called 'slosh' mechanicsm may not generate sufficient force to lengthen a syrinx. 
\\
\\
Kylstad (2014) \cite{Kyls14} simulated the viscoelastic response of the spinal cord from applied pressure and compared displacement patterns between elastic and poroelastic models. Displacement patterns were shown to be similar but differ in magnitude depending on the material parameter setting. In the most extreme cases, displacement magnitudes were approximately 15 times larger for viscoelasticity, and a lag of 10 ms were observed compared to elastic models. These are both effects that can be triggered by altering parameters for the elastic spinal cord model. St{\o}verud et al. (2015) \cite{Stov15} used poroelastic and viscoelastic models to describe the spinal cord, but to the authors knowledge poroelastic flow in the spinal cord have not yet been coupled to the fluid flow in the SAS. 
\\
(Gabriela?)
\\
\\
The goal of the study is to model CSF flow around the cord and inside the syrinx by coupling computational models of fluid flow in the SAS with elastic and poroelastic models for the spinal cord. Changes in pressure and velocity distributions in the SAS as well as cord displacements, due to the presence of a syrinx will also be investigated. 
\section{Outline}
The field of biomechanics requires multidisciplinary knowledge within medicine, mathematics, mechanics and numerical modeling. The authors main field of study as well as the main focus in this thesis lie in the last two. 
\\
In chapter 2, necessary background information to understand the physical problem from the physiological point of view is presented. The mathematical description of the physical problem based on laws of classical mechanics is given in chapter 3. The ALE-formulation, a neccessary abstraction from the physical problem is also presented in this chapter. Chapter 4 gives an introdution to the FEniCS software including a few examples validating the CFD-solver used in the thesis. In chapter 5, the implementation of a FSI model is described and validated by comparing results to a benchmark problem. Implementation of the poroelastic model is given in chapter 6, and has been devoted a chapter itself due to differences in notation. Chapter 7 gives a justification of the material parameters used in simulations of the SAS--spinal cord--syrinx system presented in chapter 8. In chapter 9, discussion, limitations, summary and possible future work is given. 
